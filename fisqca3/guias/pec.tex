\documentclass{../guias}

\usepackage{tikz}
%\usepackage[utf8x]{inputenc}

\usetikzlibrary{shapes.geometric, arrows, positioning}
\newcommand*\Eval[3]{\left.#1\right\rvert_{#2}^{#3}}
\newcommand{\pecf}[1]{\sqrt{\frac{2}{l}}\sin\left(\frac{#1\pi x}{l}\right)}
\newcommand{\pecsq}[1]{\frac{2}{l}\sin^2\left(\frac{#1\pi x}{l}\right)}
\newcommand{\pece}[1]{\frac{#1^2\hbar^2}{8\pi l^2}}

\author{Dr. Stefan Vogt}
\title{Partícula en una Caja}
\course{Físicoquímica 3}

\begin{document}
\guia

\problem{} 
Determine la magnitud de los siguientes números complejos:
\begin{enumerate}
\item $8+5i$
\item $2-2i$
\item $10+i$
\item $0.1-0.5i$
\end{enumerate}

\problem{}
Resuelva el problma de valor inicial: $y^{''} + y^{'} -6y = 0$ con las 
condiciones iniciales $y(0)=1$$y^{'}(0)=0$

\problem{}
Normalice la siguiente función de onda:
\begin{equation*}
    \psi(x) = Ae^{- \frac{1}{2}x^2}
\end{equation*}

\problem{}
Calcule la probabilidad que una partícula en una caja 1-D, 
de largo $l = 1.0 \AA$ y en el estado electrónico $n = 4$ se 
encuentre entre $l/3 \geq x \geq l/2$.

\problem{}
Cuando una partícula de masa $9.1\text{x}10^{-28}g$ en una cierta caja 1D pasa
del estado $n=5$ al estado $n=2$ emite un fotón de frecuencia $6.0\text{x}10^{14}s^{-1}$.
?`Cuál es el largo de la caja?

\problem{} Considere un electrón en una caja 1-D de largo 0.1nm.
Encuentre la frecuencia y longitud de onda de un fotón emitido
cuando el electrón
\begin{enumerate}
   \item pasa del estado $n=3$ al $n=2$
   \item pasa del estado $n=4$ al $n=2$
\end{enumerate}
?`Que fotón tiene mayor longitud de onda?

\problem{}
La frecuencia de absorción de $n=1$ a $n=2$ de para una partícula en una caja 1D es $6.0\text{x}10^{12}s^{-1}$.
Encuentre frecuencia de absorción de $n=2$ a $n=3$, para este sistema.

\problem{}
Una partícula se encuentra en el estado fundamental de una 
caja 1D de 20nm. Calcule la probabilidad que la partícula 
se encuentre entre los siguientes valores:
\begin{enumerate}
    \item $x = 9.95$ y $10.0$ nm
    \item $x = 2.95$ y $3.05$ nm
    \item $x = 19.90$ y $20.00$ nm
    \item En la primera mitad derecha
    \item En el tercio del medio
\end{enumerate}

\problem{} Calcule la energía para los primeros tres estados de un electrón y de un protón
en una caja de longitud $5.00$nm y en una de $50.0$nm.

\problem{}.
Un partícula de masa m se mueve en una caja 1D de largo $l$, con
fronteras en $x=0$ y $x=l$. Las funciones de onda normalizadas 
están dadas por: 
\begin{equation*}
    \psi(x) = \sqrt{\frac{2}{l}}sen\Big(\frac{n\pi x}{l}\Big)
\end{equation*}
con energía:
\begin{equation*}
    E_{n} = \frac{n^2h^2}{8ml^2}
\end{equation*}
Calcule la probabilidad que la partícula sea encontrada en la región $0 \geq x \geq \frac{l}{4}$.

\problem{} Un estudiante 1D que obtuvo una nota muy mala en un control. Es tirado a un pozo de longitud 5m como castigo y para 
           que pueda estudiar sin distracciones. Al tratar de aprender la materia camina  de un lado a otro
           a velocidad constante de 1.5 m/s. ?` En que estado cuántico se encuentra?

\problem{} La incertidumbre de una partícula en una caja 1D es el largo de la caja. Calcule el nivel de energía mínimo aproximado
en el cual se puede encontrar la partícula. ¿Por qué la partícula no se puede encontrar en reposo como en la mecánica clásica?  \\


\end{document}
